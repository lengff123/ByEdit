\documentclass{article}
\usepackage{amsmath}
\usepackage{algorithm}
\usepackage{algpseudocode}
\usepackage{graphicx}
\usepackage{hyperref}
\usepackage{cleveref}

\title{An Analysis of Mathematical Algorithms and Their Applications}
\author{John Smith}
\date{\today}

\begin{document}
\maketitle

\begin{abstract}
This paper comprehensively analyzes fundamental mathematical algorithms and their practical applications. We explore critical mathematical formulas and demonstrate their implementation through pseudocode algorithms. The focus is on establishing clear connections between theoretical foundations and computational approaches.
\end{abstract}

\section{Introduction}
Mathematical algorithms form the backbone of modern computational methods. This paper examines the relationship between mathematical formulas and their algorithmic implementations, using specific examples to illustrate key concepts.

\section{Theoretical Framework}
\subsection{The Quadratic Formula}
One of the most fundamental equations in algebra is the quadratic formula, which provides solutions to quadratic equations of the form $ax^2 + bx + c = 0$. The formula is given by:
\[
x = \frac{-b \pm \sqrt{b^2 - 4ac}}{2a}
\]

This formula has widespread applications in various fields, including physics, engineering, and computer graphics.

\section{Algorithmic Implementation}
\subsection{Recursive Factorial Calculation}
To demonstrate the implementation of mathematical concepts in algorithms, we present a recursive approach to calculating factorials:

\begin{algorithm}
\caption{Recursive Factorial Calculation}
\begin{algorithmic}[1]
\Function{Factorial}{$n$}
    \If{$n = 0$}
        \Return 1
    \EndIf
    \State \Return $n \times$ \Call{Factorial}{$n-1$}
\EndFunction
\end{algorithmic}
\end{algorithm}

This algorithm demonstrates the elegant relationship between mathematical recursion and computational implementation.

\section{Discussion}
The presented algorithm showcases how mathematical concepts can be translated into efficient computational procedures. The recursive approach provides both mathematical elegance and practical utility.

\section{Conclusion}
This paper has demonstrated the synergy between mathematical formulas and algorithmic implementations. Future work could explore more complex algorithms and their applications in real-world scenarios.

\bibliographystyle{plain}
\begin{thebibliography}{9}
\bibitem{knuth1973art}
Knuth, D. E. (1973). The Art of Computer Programming, Vol. 1: Fundamental Algorithms. Addison-Wesley.

\bibitem{cormen2009introduction}
Cormen, T. H., Leiserson, C. E., Rivest, R. L., \& Stein, C. (2009). Introduction to Algorithms (3rd ed.). MIT Press.
\end{thebibliography}

\end{document}